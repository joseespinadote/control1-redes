\documentclass{article}
\usepackage[utf8]{inputenc}
\usepackage[spanish]{babel}
\usepackage{graphicx}
\usepackage{hyperref}
\usepackage[ruled,vlined]{algorithm2e}

\graphicspath{{C:/Users/josee/Documents/mag/redes/control1-redes/imagenes/}}

\begin{document}

\title{Control 1\\CC4303-2 Redes}
\author{José Espina\\joseguillermoespina@gmail.com}
\date{}
\maketitle
\section{Contexto, problema y requerimientos}
Uno de los grandes problemas que ha aparecido con la movilidad de usuarios es la pérdida de conexión entre cliente y servidor ya sea por cambio de antena, paso de internet móvil a wi-fi, de wi-fi a cable, y viceversa. Esto lleva a que se sobrecargue la inteligencia por el lado del servidor (si se usa UDP) o se cierren/abran conexiones con el consiguiente overhead en handshaking (si se usa TCP). Algunos ingenieros señalan que la capa de transporte no tiene por qué preocuparse de eso y debiesen ser las capas superiores las encargadas. En esta actividad se le pedirá diseñar o modificar un protocolo de comunicaciones para este nuevo escenario

A partir de las instrucciones, se identifican los siguientes requerimientos:
\begin{enumerate}
\item Pruebe su solución construyendo un servidor http que implemente GET, recibiendo peticiones de un cliente “móvil” (puede construir una función UniqID() para identificar cada dispositivo) y le entregue a cada dispositivo su respuesta, sin repetir ni equivocarse
\item El cliente debiese poder conectarse al servidor, pedir un archivo, recibir parte de él, desconectarse, conectarse de nuevo, pedir un archivo y recibir el resto
\item Su solución debe escribirse en python usando sockets, sin otras librerías para redes y NO DEBE caerse cuando el servidor le responde a un cliente ``inexistente" (que se está moviendo)
\item La entrega serán dos archivos: un .zip con su solución en python de su cliente y servidor (código fuente), y un PDF donde estará un pequeño informe con sus decisiones de diseño, supuestos seguidos (que sean razonable, no pueden suponer que sólo se conectarán una vez, por ejemplo)
\item Responda justificadamente a las siguientes preguntas:
\begin{enumerate}
\item ¿Qué pasa cuando el cliente termina de recibir un archivo pero igual lo sigue pidiendo? ¿Por qué?
\item ¿Cómo funciona su protocolo si hay múltiples clientes? digamos que su servidor tiene la misma tasa de consultas que Google.
\item A partir de la pregunta anterior ¿Cambia su respuesta si en vez de texto responde con multimedia (tamaño > 1 MB)? ¿Por qué?
¿Su solución sigue siendo válida si se usan proxies http? ¿Por qué?
\end{enumerate}
\end{enumerate}

\section{Propuesta de protocolo}

Se propone la creación de un pequeño protocolo que funcionará en las capas superiores a la de transporte, usando TCP. Se implementa un prototipo a través del \textit{framework} socketserver de Python 3\cite{web_python3_socketserver}. La funcionalidad básica del software se representa con el siguiente diagrama de flujo y algoritmo

\begin{figure}[hp]
\centering
\includegraphics[scale=0.4]{diagramaflujo}
\caption{Diagrama de flujo de la solución propuesta}
\end{figure}

\begin{algorithm}[H]
\SetAlgoLined
\KwIn{Solicitud de un cliente mediante socket abierto}
\KwOut{Nuevo identificador de cliente, ó ``chunks" de datos de archivo (texto o binario) al cliente}
 inicializacion variables y constantes\;
 request = socket.recv(SIZE)\;
 \eIf{type(request) != GET}{
 	socket.send(ERROR\_400)\;
 }{
   \eIf{id\_cliente in request}{
     file.seek(progresos[id\_cliente])\;
     data = file.read(CHUNK\_SIZE)\;
     socket.send(OK\_200)\;
     \While{data}{
       socket.send(data)\;
       progresos[id\_cliente] += len(data)\;
       data = file.read(CHUNK\_SIZE)\;
     }
   }{
     id\_cliente=uuid.uuid4()\;
     progresos[id\_cliente] = 0\;
     socket.send(OK\_200)\;
     socket.send(id\_cliente)\;
   }
}
\caption{Prototipo mini-protocolo}
\end{algorithm}

Las razones de la elección de TCP sobre UDP en la capa de transporte son las siguientes

\begin{enumerate}
\item El servidor enviará archivos texto plano y binarios. Es importante que los paquetes lleguen íntegros, y UDP no lo asegura
\item No se requiere hacer \textit{broadcast} de paquetes
\item La solución usa parte de HTTP para confeccionar la respuesta, y en general HTTP se usa sobre TCP por ser un protocolo confiable\cite{rfc_2616}
\end{enumerate}

\section{La implementación}

Se implementó un \textit{handler} básico para cumplir con los requerimientos nombrados en la primera sección El código se adjunta con la tarea y, también, está disponible en un repositorio privado en GitHub (\url{https://github.com/joseespinadote/control1-redes}, solicitar acceso al autor). Los partes fundamentales de la implementación se explican a continuación:






\cite{libro_redes_python}Libro
\begin{thebibliography}{9}
\bibitem{web_python3_socketserver} Documentación oficial del \textit{framework} socketserver de Python 3 \url{https://docs.python.org/3/library/socketserver.html}, visitado durante Mayo del 2020
\bibitem{rfc_2616} RFC 2616 \url{https://www.ietf.org/rfc/rfc2616.txt}
\bibitem{libro_redes_python} Rhodes, B., \& Goerzen, J. (2010). Foundations of Python network programming: the comprehensive guide to building network applications with Python. New York: Apress.
\end{thebibliography}
\end{document}